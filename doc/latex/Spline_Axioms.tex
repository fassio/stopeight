\documentclass[a4paper,landscape]{report}
\usepackage{amsfonts}
\usepackage{amsmath}
\usepackage{amssymb}
\usepackage{hyperref}

\iffalse
\usepackage{biblatex}
\addbibresource{Stopeight.bib}
\fi
\usepackage{natbib}

\renewcommand{\baselinestretch}{1.25}
\begin{document}
\title{Spline Axioms}
\author{Fassio Blatter}
\maketitle

\chapter{Single Spline}

\subsection{Affine Transformations}
The matrix $M$ is a product of affine (sub-)space transformations. Any set of vectors $h'$ within bounds $[a,b]$ can be translated, rotated and scaled, so that $h'(a)=\begin{pmatrix}0 & 0\end{pmatrix}$ and $h'(b)=\begin{pmatrix}1 & 0\end{pmatrix}$. Any approximation with a parametrisation $t$ can be expressed with a single spline if the Coefficients can be brought into a triangular form.\\\\
The simple case of a straight line
\begin{equation}
line(t):
\begin{pmatrix}
1 & t
\end{pmatrix}
\underbrace{\begin{bmatrix}
1 & 0\\
-1 & 1
\end{bmatrix}}_{Coefficients}
M
\begin{pmatrix}
h_{1} \\
h_{2}
\end{pmatrix}
=
\begin{pmatrix}
h_{1}' & h_{2}'
\end{pmatrix}
\end{equation}
The more complicated case of a quadratic bezier spline
\begin{equation}
quad(t):
\begin{pmatrix}
1 & t & t^2
\end{pmatrix}
\underbrace{\begin{bmatrix}
1 & 0 & 0\\
-1 & 1 & 0\\
0 & -1 &1
\end{bmatrix}}_{Coefficients}
M
\begin{pmatrix}
h_{1} \\
h_{2} \\
h_{3}
\end{pmatrix}
=
\begin{pmatrix}
h_{1}' & h_{2}' & h_{3}'
\end{pmatrix}
\end{equation}
The Coefficients are composed of the sum of identity matrices in lower left sub-spaces.
The characteristic polynomials of the diagonal and bidiagonal is composed of binomial coefficients.
\begin{equation}
\chi_{\begin{pmatrix}1 & 0 & 0\\0 & 1 & 0\\0 & 0 & 1\end{pmatrix}} = \begin{pmatrix}3 \\ 0\end{pmatrix}*t^0 - \begin{pmatrix}3 \\ 1\end{pmatrix}*t^1 + \begin{pmatrix}3 \\ 2\end{pmatrix}*t^2 - \begin{pmatrix}3 \\ 3\end{pmatrix}*t^3;
\chi_{\begin{pmatrix}-1 & 0\\0 & -1\end{pmatrix}} = \begin{pmatrix}2 \\ 0\end{pmatrix}*t^0 + \begin{pmatrix}2 \\ 1\end{pmatrix}*t^1 + \begin{pmatrix}2 \\ 2\end{pmatrix}*t^2
\end{equation}
The quartic spline
\begin{equation}
quart(t):
\begin{pmatrix}
1 & t & t^2 & t^3 & t^4
\end{pmatrix}
\underbrace{\begin{bmatrix}
1 & 0 & 0 & 0 & 0\\
-1 & 1 & 0 & 0 & 0\\
0 & -1 &1 & 0 & 0\\
0 & 0 & -1 & 1 & 0\\
0 & 0 & 0 & -1 & 1
\end{bmatrix}}_{Coefficients}
M
\begin{pmatrix}
h_{1} \\
h_{2} \\
h_{3} \\
h_{4} \\
h_{5}
\end{pmatrix}
=
\begin{pmatrix}
h_{1}' & h_{2}' & h_{3}' & h_{4}' & h_{5}'
\end{pmatrix}
\end{equation}
Note: Because we are only interested in a $2x2$ subspace, the tri- and quadridiagonals can be ignored. It is a polynomial of degree 4 in a 2 dimensional subspace. $h'_{3},h'_{4},h'_{5}$ are irrelevant in this case.\\\\
Axiom: Because of associativity, an inverse of the control points can be left multiplied.
\begin{equation}
quart(t):
\begin{pmatrix}
1 & t & t^2 & t^3 & t^4
\end{pmatrix}
\underbrace{\begin{bmatrix}
1 & 0 & 0 & 0 & 0\\
-1 & 1 & 0 & 0 & 0\\
0 & -1 &1 & 0 & 0\\
0 & 0 & -1 & 1 & 0\\
0 & 0 & 0 & -1 & 1
\end{bmatrix}}_{Coefficients}
M
=
H^{-1}
\begin{pmatrix}
h_{1}' & h_{2}' & h_{3}' & h_{4}' & h_{5}'
\end{pmatrix}
\end{equation}

\iffalse
\subsection{Number of Variables}
Axiom: Within the bounds $[a,b]$ a combination of multiple $splines$ can occur. They are independent, $iff$ for all vectors there is a sum $g' +h'$\\\\
\begin{equation}
sum(t):
\begin{bmatrix}
1 & t & t^2\\
1 & u & u^2
\end{bmatrix}
\underbrace{CMH}_{}
=
\begin{pmatrix}
g_{1}' & g_{2}' & g_{3}'\\
h_{1}' & h_{2}' & h_{3}'
\end{pmatrix}
\end{equation}
Theorem: They are harmonic, hence there is a non-sparse separation.
\fi

\subsection{Spline Base}
\subsubsection{Existence of Bounds}
A base spline can be found for the integer $Image \in \mathbb{Z}^3$ in dimensions $x,y$
\begin{equation}
base(t):
\begin{pmatrix}
1 & t & t^2
\end{pmatrix}
\underbrace{\begin{pmatrix}
1 & 0 & 0\\
-1 & 1 & 0\\
0 & -1 & 1
\end{pmatrix}\begin{bmatrix}
M\begin{pmatrix}0 \\ 0 \\ 0\end{pmatrix} & Mc_{1} & M \begin{pmatrix}1 \\ 0 \\ 0\end{pmatrix}
\end{bmatrix}}_{Base}
=
\underbrace{
\begin{pmatrix}
h_{1}' & h_{2}' & h_{3}'
\end{pmatrix}}_{Image}
\end{equation}
For a Lie Group
\begin{align}
\text{\{GL(3,$\mathbb{Z}$)} \subset M \vert \forall m \in M \exists c_{1}\}
\end{align}
Axiom: For a spline base with $h_{1}=\begin{pmatrix}0 \\ 0 \\ 0\end{pmatrix}$ and $h_{2}=\begin{pmatrix}1 \\ 0 \\ 0\end{pmatrix}$, there exist only two Eigenvalues ~\cite["Spline\_Eigensystems.nb"]{Stopeight}
\begin{align}
ev_{1}=Eigenvalue(
\underbrace{\begin{pmatrix}
a & 0 & 0\\
b & c & 0\\
0 & d & e
\end{pmatrix}}_{Base})\\
ev_{2}=Eigenvalue(
\underbrace{\begin{pmatrix}
0 & 0 & 0\\
b & 0 & 0\\
0 & d & 0
\end{pmatrix}}_{Base})
\end{align}
Theorem: The approximation algorithm imposes a constraint on the Eigenvectors ~\cite["Spline\_EVConstraint.nb"]{Stopeight}

\chapter{Polynomial Degree}
\subsection{Combination of Spaces}
A quartic bezier spline has a total of five Control points which can be reduced to three if the subspline alignment is smooth. A circle has a total of two focal points, which can be reduced to one if smoothness applies. If we deviate from this smoothness, a Spiral forms.
\subsection{Parametrised Korners}
Beyond quadratic approximation, each Corner is a Korner $K$ with a focal vector. On a Spiral there are multiple Korners, on a ZigZag, there is none. The Korners mostly make sense in the context of sub-pulse analysis, i.e. the comparison of rise and fall.\\\\
If we have $T_{3}$ and all the Compact Covers worked out, Korners can be defined (Swell?) and new differences and areas parametrised. $p$ supremums are considered over the whole section of the Compact Cover $V_{n}$. The interval over the bounds of a supremum is $[a(K_{p}),b(K_{p})]$.
\begin{align}
K_{p+1} = \{ C_{3} \in  [a,b]\vert \sup \limits _{V_{n} \setminus [a(K_{p}),b(K_{p})]} \lvert curvature(a,b) \rvert \}
\end{align}
There is one Korner per arc $\{U_{m}\}$.\\
As a consequence $diameter$ and $area$ could be parametrised on the new Korners $dom(\iota)=T_{3}\cup K_{p}$. ($dom(\xi)$?).\\
\begin{align}
area_{sup}(K_{p})=\int \limits _{a(K_{p})}^{b(K_{p})} \lvert \iota_{K_{p}}(t)-\iota_{K_{p+1}}(t) \rvert \mathrm{d}t\\
diameter_{sup}(a,b)=\sup \limits _{c,d \in [a,b]} \lvert \xi(d) - \xi(c) \rvert
\end{align}
The solid angle now depends on the bisection and $area_{sup}$. (sum or product?)
\begin{equation}
solidangle_{Abs}(a,b)=\sum \limits _{i=0}^{p} \frac{area_{sup}(K_{i})}{(\xi(t(K_{i}))-\iota(t(K_{i})))^2}
\end{equation}
The horizontals on the bisections are defined by the transversality of the eliptic approximation.
\begin{align}
area_{horiz}(a,b)=\int \limits _{0}^{?} \lvert \xi(b-t)-\xi(a+t) \rvert \mathrm{d}t
\end{align}
(change of variable?)
\subsection{Tangential Paraglide Differences}
Focal points are intersections of normals on tangents. Distribution can be parametrised without breaking the bounds $iff$ there is only one Korner. Adjusted $area$

\iffalse
\printbibliography
\fi
\bibliography{Stopeight}{}
\bibliographystyle{plain}

\end{document}