\documentclass[a4paper,portrait]{report}
\usepackage{amsfonts}
\usepackage{amsmath}
\usepackage{amssymb}

\iffalse
\usepackage{biblatex}
\addbibresource{Stopeight.bib}
\fi
\usepackage{natbib}

\renewcommand{\baselinestretch}{1.25}
\begin{document}
\title{Spline Axioms}
\author{Fassio Blatter}
\maketitle

\chapter{Spline Approximations}
A set of vectors $\begin{pmatrix}x \\ y\end{pmatrix} \in X$ within any parametrisation bounds $t\in[a,b]$ can be translated, rotated and scaled by $M^{3x3}$, so that $\iota_{X}(a)=\begin{pmatrix}0 \\ 0\end{pmatrix}$ and $\iota_{X}(b)=\begin{pmatrix}1 \\ 0\end{pmatrix}$. An approximation (arc) $A$ of arbitrary polynomial degree is required.\\\\
For example, approximating a straight line has polynomial degree one.
\begin{equation}
line(t):
\begin{pmatrix}
1 \\
t
\end{pmatrix}
\underbrace{\begin{pmatrix}
1 & -1\\
0 & 1
\end{pmatrix}}_{Jordanian}
(\underbrace{\begin{bmatrix}
1 & 0\\
0 & 1
\end{bmatrix}}_{M^{2x2}}
\begin{pmatrix}
0 & 1\\
0 & 0
\end{pmatrix})
=
\begin{pmatrix}
x \\
y
\end{pmatrix}
\end{equation}
Note: Because $S$ and $E$ are already at $\begin{pmatrix}0\\0\end{pmatrix}$ and $\begin{pmatrix}1\\0\end{pmatrix}$, the matrix $M^{2x2}$ of affine transformations is the identity matrix.\\\\
The more complicated case of a quadratic bezier spline
\begin{equation}
quad(t):
\begin{pmatrix}
1 \\
t \\
t^2
\end{pmatrix}
\underbrace{\begin{pmatrix}
1 & -1 & 0\\
0 & 1 & -1\\
0 & 0 &1
\end{pmatrix}}_{Jordanian}
(M^{3x3}*\underbrace{\begin{pmatrix}
x(S) & x(C_{1}) & x(E) \\
y(S) & y(C_{1}) & y(E) \\
0 & 0 & 0
\end{pmatrix}}_{\text{Bezier Control Points}})
=
\begin{pmatrix}
x \\
y \\
0
\end{pmatrix}
\end{equation}
Note: $M^{NxN}$ has Rank 2\\\\
The Jordanian is composed of the sum of identity matrices in upper right sub-spaces.
The characteristic polynomials of the diagonal and bidiagonal are composed of binomial coefficients.
\begin{align}
\chi_{\begin{pmatrix}1 & 0 & 0\\0 & 1 & 0\\0 & 0 & 1\end{pmatrix}} = \begin{pmatrix}3 \\ 0\end{pmatrix}*t^0 - \begin{pmatrix}3 \\ 1\end{pmatrix}*t^1 + \begin{pmatrix}3 \\ 2\end{pmatrix}*t^2 - \begin{pmatrix}3 \\ 3\end{pmatrix}*t^3\\
\chi_{\begin{pmatrix}-1 & 0\\0 & -1\end{pmatrix}} = \begin{pmatrix}2 \\ 0\end{pmatrix}*t^0 + \begin{pmatrix}2 \\ 1\end{pmatrix}*t^1 + \begin{pmatrix}2 \\ 2\end{pmatrix}*t^2
\end{align}
Note: It is a sparse approximation.\\\\
If start and end points are limited to be in the transversality $S,E \in Y$, the maximal polynomial degree can be reduced to four.
\begin{equation}
quart(t):
\begin{pmatrix}
1 \\ t \\ t^2 \\ t^3 \\ t^4
\end{pmatrix}
\underbrace{\begin{pmatrix}
1 & -1 & 0 & 0 & 0\\
0 & 1 & -1 & 0 & 0\\
0 & 0 &1 & -1 & 0\\
0 & 0 & 0 & 1 & -1\\
0 & 0 & 0 & 0 & 1
\end{pmatrix}}_{Jordanian}
(M^{5x5}*H^{5x5})
=
\begin{pmatrix}
x \\ y \\ 0 \\ 0 \\ 0
\end{pmatrix}
\end{equation}
Note: Because we are only interested in planar coordinates, the tri- and quadridiagonals of the Jordanian can be ignored. It is a polynomial of degree 4 in a 2 dimensional subspace (Sparse).\\\\
Axiom: Because of associativity, an inverse of the control points can be left multiplied. Because of associativity, the inverse of the Jordanian can be left multiplied.
\begin{equation}
quart(t):
\begin{pmatrix}
1 \\ t \\ t^2 \\ t^3 \\ t^4
\end{pmatrix}
=
\underbrace{\begin{bmatrix}
1 & 1 & 1 & 1 & 1\\
0 & 1 & 1 & 1 & 1\\
0 & 0 & 1 & 1 & 1\\
0 & 0 & 0 & 1 & 1\\
0 & 0 & 0 & 0 & 1
\end{bmatrix}}_{Jordanian^{-1}}
(M^{5x5}H^{5x5})^{-1}
\begin{pmatrix}
x \\ y \\ 0 \\ 0 \\ 0
\end{pmatrix}
\end{equation}

\chapter{Representation}

A quartic spline is fully defined by $S,E$ and three on spline control points $c_{1},c_{3} \in A;c_{2} \in C_{4}$. The corresponding Bezier control points $q_{1},q_{2} \in Q_{3}$ are obtained by substituting $t=1/4$,$t=2/4$ and $t=3/4$ with $c_{1}$,$c_{2}$ and $c_{3}$ in the quartic linear equation.
\begin{align}
H=\begin{bmatrix}
x(S) & x(q_{1}) & x(C_{4}) & x(q_{2}) & x(E) \\
y(S) & y(q_{1}) & y(C_{4}) & y(q_{2}) & x(E) \\
0 & 0 & 0 & 0 & 0\\
0 & 0 & 0 & 0 & 0\\
0 & 0 & 0 & 0 & 0
\end{bmatrix}\\
representation: S \times C_{4} \times E \rightarrow A \times V \times A\\
S,C_{4},E \in A; S,V,E \in \iota_{T}\\
(S,C,E)\mapsto(c_{1},v_{1},c_{3})
\end{align}\\
The four quadratic splines $\gamma_{H_{1}} ... \gamma_{H_{4}}$ are defined by $S,E$ and on spline control points $c_{1},c_{2},c_{3},c_{5},c_{6},c_{7} \in A;c_{4} \in C_{3}$. The corresponding Bezier control points $q_{1},q_{2},q_{3},q_{4} \in Q_{1}$ for the pieces $H_{1},H_{2},H_{3},H_{4}$ are obtained by substituting $t=0,t=1/2$ and $t=1$ with $c_{1},c_{3},c_{5},c_{7}$ in each of the four quadratic linear equations.
\begin{align}
H_{1}=
\begin{bmatrix}
x(S) & x(q_{1}) & x(c_{2})\\
y(S) & y(q_{1}) & y(c_{2})\\
0 & 0 & 0
\end{bmatrix};
H_{2}=
\begin{bmatrix}
x(c_{2}) & x(q_{2}) & x(c_{4})\\
y(c_{2}) & y(q_{2}) & y(c_{4})\\
0 & 0 & 0
\end{bmatrix}\\
H_{3}=
\begin{bmatrix}
x(c_{4}) & x(q_{3}) & x(c_{6})\\
y(c_{4}) & y(q_{3}) & y(c_{6})\\
0 & 0 & 0
\end{bmatrix};
H_{4}=
\begin{bmatrix}
x(c_{6}) & x(q_{4}) & x(E)\\
y(c_{6}) & y(q_{4}) & y(E)\\
0 & 0 & 0
\end{bmatrix}
\end{align}
\section{Geometric Meaning}
The axis connecting $C_{4}$ to $V$ is the focal axis in a perfect half circle. In an oval which can not be ruled out, $C_{4}$ is the only point intersecting the focal axis with a tangent on-spline. If the $S_{2}\cap E_{2}$ are bent outwards from the perfect half circle position, there are a total of three focal axes in a quartic spline, which is the default for a quartic spline. Because the Stopeight\_Analyzer algorithm is catching the bends starting from the limit of $straightness$, this case can be ruled out. Therefore, the number of $card(C_{4})$ per chart $U_{m}$ is reduced from $3$ to $1$.

\chapter{Inverse Representation}
In Representation\_Parallelogram.nb and Representation\_Circles.nb there are currently two representations evaluated. The parallelogram construction favors half-circles, while the more computationally extensive circles positions the two omitted $C_{4}$ closer togther for ovals, etc. The latter is described below.\\\\
The inverse of the representation reconstructs $S,C_{4},E$ from the left and right on-spline control points $q \in A_{4}$ and the intersection of the angle-bisection in the triangle $SCE$ with the line $SE$.
\begin{align}
representation^{-1}: A \times V \times A \rightarrow S \times C_{4} \times E\\
(c_{1},v_{1},c_{3})\mapsto(S,C_{4},E)
\end{align}
The point $C_{4}$ is relatively easy to reconstruct under the assumption, that the line...\\\\
\section{Geometric Meaning}
If the representation itself is re-interpreted to define a quartic spline of its own $S=c_{1},C_{3}=v_{1},E=c_{3}$, the resulting geometric figure is a sort of reflection surface, with the area narrowing down when repeating this procedure. The focal axis of each subsequent quartic spline rotates depending on how far towards $S$ or $E$, the single $C_{4}$'s are stretched. The inverse expands this area.

\chapter{Signals}
A quartic representation is sufficient for pen-stroke analysis. Two possible implementations are provided. Going from two quartic to a quintic requires $n-3$ on-spline control points, $n$ stands for the polynomial degree. In a Spiral, it would make sense to position these at the $C_{4}$ of the two Cliffs with $t_{5}=1/3$ and $t_{5}=2/3$. For two circles, the control points would have to be at $t_{6}=1/4,t_{6}=2/4$ and $t_{6}=3/4$. The number of quartic splines required for a polynomial of higher degree doubles with every step $4...n$. The control points for a Swell would have to be at the $C_{4}$'s.\\\\
Axiom: Factorisation requires maximal polynomial degree.\\\\
Axiom: Frame-bounds have to be $S,E$ of Compact Covers.\\\\
Axiom: Changes in polynomial degree indicate third derivatives (of the approximation).\\\\
Axiom: Complex multiplication preserves signal characteristics even in time-variant scenario.\\\\
Axiom: Binomial formula apply to complex multiplication\\\\
\iffalse
\printbibliography
\fi
\bibliography{Stopeight}{}
\bibliographystyle{plain}

\end{document}
