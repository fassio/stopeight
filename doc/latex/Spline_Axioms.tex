\documentclass{report}
\usepackage{amsfonts}
\usepackage{amsmath}
\usepackage{amssymb}
\usepackage{hyperref}
\usepackage{biblatex}
\renewcommand{\baselinestretch}{1.25}

\addbibresource{Stopeight.bib}

\begin{document}
\title{Spline Axioms}
\author{Fassio Blatter}
\maketitle

\chapter{Linear Limitation}

\subsection{Affine Transformations}
If a matrix is a product of affine (sub-)space transformations, its upper triangular matrix is an expression of a limitation to the type and combination of the factors.\\\\
The simple case of a straight line
\begin{equation}
line(t):
\begin{pmatrix}
1 & t
\end{pmatrix}
\underbrace{\begin{bmatrix}
1 & 0\\
-2 & 1
\end{bmatrix}}_{Coefficients}
\begin{pmatrix}
Mh_{1}
\end{pmatrix}
=
\begin{pmatrix}
h_{1}' & h_{2}'
\end{pmatrix}
\end{equation}
The more complicate case of a quadratic bezier spline
\begin{equation}
quad(t):
\begin{pmatrix}
1 & t & t^2
\end{pmatrix}
\underbrace{\begin{bmatrix}
1 & 0 & 0\\
-2 & 2 & 0\\
1 & -2 &1
\end{bmatrix}}_{Coefficients}
\begin{pmatrix}
Mh_{1} & Mh_{2} & Mh_{3}
\end{pmatrix}
=
\begin{pmatrix}
h_{1}' & h_{2}' & h_{3}'
\end{pmatrix}
\end{equation}
The coefficient Matrix $Coefficients$ is not garanteed to be invertible. It can be a product of non-commutative factors. Or there can be illegal shearing.

\subsection{Number of Variables}
If the coefficient Matrix of a polynomial can be orthogonalised, the polynomial can be defined by a single variable. The nonlinear equation
\begin{equation}
nonlinear(t):
\begin{bmatrix}
1 & t & t^2\\
1 & u & u^2
\end{bmatrix}
\underbrace{\begin{bmatrix}
2 & -2 & 1\\
-2 & 4 & -2\\
1 & -2 & 2
\end{bmatrix}}_{Coefficients}
\begin{pmatrix}
Mh_{1} & Mh_{2} & Mh_{3}
\end{pmatrix}
=
\begin{pmatrix}
h_{1}' & h_{2}' & h_{3}'
\end{pmatrix}
\end{equation}
has the same control points, but the coefficient matrix $Coefficients$ can be singular $det(Coefficients)=0$.

\chapter{Limited Image}
If an orthonormal base can be found for the $Image \in \mathbb{Z}^3$
\begin{equation}
orthonormal(t):
\begin{pmatrix}
1 & t & t^2
\end{pmatrix}
\underbrace{\begin{pmatrix}
1 & 0 & 0\\
0 & 1 & 0\\
0 & 0 & 1
\end{pmatrix}}_{Coefficients}
\underbrace{\begin{bmatrix}
Me_{1} & Me_{2} & Me_{3}
\end{bmatrix}}_{Base}
=
\underbrace{
\lambda
\begin{bmatrix}
Me_{1}' & Me_{2}' & Me_{3}'
\end{bmatrix}}_{Image}
\end{equation}
there do not exist any polynomials other than the one expressed by the product of the coefficient matrix $Coefficients$ and the $Base$. It is a (diffeo?)morphism.\\\\
For a Lie Group a characteristic polynomial can be found
\begin{align}
orthonormal_{Coefficients,Base}(\lambda): \text{GL(3,R)} \rightarrow Polynomial\\
GL(3,R) = ONB(a,b); [a,b] \subseteq \text{Chart of }Domain
\end{align}

\chapter{Eliptic Limitation}
\subsection{Combination of Bases}
A quartic bezier spline has a total of five Control points which can be reduced to three if the subspline alignment is smooth. A circle has a total of two Focal Points, which can be reduced to one if smoothness applies. If we deviate from this smoothness, a Spiral forms.
\subsection{Parametrised Focal Points}
Beyond quadratic approximation, each Corner is a Focal point with a focal vector. On a Spiral there are multiple Corners, on a ZigZag, there is none. The Focal Points mostly make sense in the context of sub-pulse analysis, i.e. the comparison of rise and fall.\\\\
If we have $T_{3}$ and all the Compact Covers worked out, Focal points can be defined (Swell?) and new differences and areas parametrised. $p$ supremums are considered over the whole section of the Compact Cover $V_{n}$. The interval over the bounds of a supremum is $[a(F_{p}),b(F_{p})]$.
\begin{align}
F_{p+1} = \{ C_{3} \in  [a,b]\vert \sup \limits _{V_{n} \setminus [a(F_{p}),b(F_{p})]} \lvert curvature(a,b) \rvert \}
\end{align}
There is one Focal point per arc $\{U_{m}\}$.\\
As a consequence $diameter$ and $area$ could be parametrised on the new Focal points $dom(\iota)=T_{3}\cup F_{p}$. ($dom(\xi)$?).\\
\begin{align}
area_{sup}(K_{p})=\int \limits _{a(F_{p})}^{b(F_{p})} \lvert \iota_{F_{p}}(t)-\iota_{F_{p+1}}(t) \rvert \mathrm{d}t\\
diameter_{sup}(a,b)=\sup \limits _{c,d \in [a,b]} \lvert \xi(d) - \xi(c) \rvert
\end{align}
The solid angle now depends on the bisection and $area_{sup}$. (sum or product?)
\begin{equation}
solidangle_{Abs}(a,b)=\sum \limits _{i=0}^{p} \frac{area_{sup}(F_{i})}{(\xi(t(F_{i}))-\iota(t(F_{i})))^2}
\end{equation}
The horizontals on the bisections are defined by the transversality of the eliptic approximation.
\begin{align}
area_{horiz}(a,b)=\int \limits _{0}^{?} \lvert \xi(b-t)-\xi(a+t) \rvert \mathrm{d}t
\end{align}
(change of variable?)
\subsection{Tangential Paraglide Differences}
Depend on focal points. Adjusted $area$

\printbibliography

\end{document}