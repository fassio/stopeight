\documentclass{article}
\usepackage{amsfonts}
\usepackage{amsmath}
\usepackage{hyperref}
\usepackage{biblatex}
\renewcommand{\baselinestretch}{1.25}

\addbibresource{Stopeight_Analyzer.bib}
\iffalse
\bibliography{Stopeight_Analyzer}{}
\bibliographystyle{plain}
\fi

\begin{document}
\title{Stopeight Analyzer}
\author{Fassio Blatter}
\maketitle

\section{Introduction}

The algorithm in this text has been developed during the years 2009 to 2016 by Specific Purpose Software GmbH. It is now open sourced, but mostly exists in the form of code fragments.\\
This file is in the Stopeight repository on Github. Please edit here:\\
\href{https://github.com/specpose/stopeight/tree/master/doc/latex/Stopeight_Analyzer.tex}{https://github.com/specpose/stopeight/tree/master/doc/latex/Stopeight\_Analyzer.tex}
A brief overview can be found here:\\
\href{https://www.stopeight.com/dev_analyzer.html}{https://www.stopeight.com/dev\_analyzer.html}\\\\
The Arcs $A \subset \mathbb{R}^2$ are an approximation of a set$\footnote{Midpoints have to be formally included for insertion only}$ $X \subset \mathbb{R}^2$:
\begin{align}
f \circ g: X \rightarrow A
\end{align}
\begin{equation*}
1 < n < \infty
\end{equation*}
via the intermediate functions $f,g$ where intermediate control set $Y$ defines Arcs $U_{m}$. ($f \circ g$ is neither injective nor surjective) ~\cite[\nopp 2.1]{LauresSzymik}:
\begin{align}
\iffalse
Y = \{g^{-1}U \mid U \text{ open in } A\}\\
\fi
f: X \rightarrow Y; g: Y \rightarrow A
\end{align}
Note: The set of Arcs A is a manifold. In $A$ we can find a system of open sets. The set of charts $\{U_{m}\}$ covering the whole manifold $A$ is an atlas. ($X_{m}$ is a 0-manifold, homeomorphism?)~\cite[\nopp 3.1.1.]{Fomenko}
\begin{equation}
A = \cup_{m}U_{m}
\end{equation}
Within $f$ we are assigning a sequence of three points to compact (= closed and bounded) Hausdorf invervals:
\begin{equation}
\{x_{n}\}_{n \in \mathbb{N}} \mapsto \{s,c,e\}
\end{equation}\\
Each element $y \in Y$ is composed of a start of Turn $s \in X$, a Corner$\footnote{insertion: $c \notin X$}$ $c \in X$ and an end of Turn $e \in X$. This format has the benefit of reducing the amount of data for $Computation$ (see Stopeight Comparator), while preserving the $Representation$ using quadratic bezier splines.

\section{Representation}

The affine transformations $q$ are functions that map to control points $Q \subset C \subset \mathbb{R}^2$ which do not lie in the original set $X$ (1):
\begin{align}
q: Y \rightarrow Q\\
Q \cap X = 0; Q \cap Y = 0
\end{align}
which define subsegments $\{c_{1}, ... , c_{4}\} \in S \subset U_{m} \in A$ in combination with $Y \subset C, X, U_{m}$:
\begin{align*}
Q,Y \subset C
\end{align*}
\begin{align*}
c_{1}=\{s,q_{1}(s,c),q_{2}(s,c)\}\\
c_{2}=\{q_{2}(s,c),q_{3}(s,c),c\}\\
c_{3}=\{c,q_{4}(c,e),q_{5}(c,e)\}\\
c_{4}=\{q_{5}(c,e),q_{6}(c,e),e\}
\end{align*}
The Arc segment defining function $qbs$ is composed of four quadratic bezier splines and forms the coordinates $a=\{x',y'\}$ of Arc $U_{m}$:
\begin{equation}
qbs: S \rightarrow A, \{c_{1},c_{2},c_{3},c_{4}\} \mapsto \{a_{m}\}_{m \in \mathbb{N}}
\end{equation}
Note: The continuous function $qbs$ makes the topological space $A$ path-connected. Therefore it is a $path$ from $a_{m}$ to $a_{m+1}$ and intersects at least once $dim (U_{m} \cup U_{m+1}) \geq 1$. ~\cite[\nopp 6.1.3.]{Mortad}\\\\
(Arcs $A$ has a euclidean metric $d$:?)\\\\\\\\
(which makes $(A,d)$ a topology?)\\\\\\\\

\section{Computation}

\iffalse
\begin{equation} 
\forall u,v \in V :
d(u,v) = 
\begin{cases}
0,  u=v \\
1,  u \neq v 
\end{cases}
\end{equation}
\fi

\printbibliography

\end{document}