\documentclass{article}
\usepackage{amsfonts}
\usepackage{amsmath}
\usepackage{hyperref}
\usepackage{biblatex}
\renewcommand{\baselinestretch}{1.25}

\addbibresource{Stopeight_Analyzer.bib}
\iffalse
\bibliography{Stopeight_Analyzer}{}
\bibliographystyle{plain}
\fi

\begin{document}
\title{Stopeight Analyzer}
\author{Fassio Blatter}
\maketitle

\section{Introduction}

The algorithm in this text has been developed during the years 2009 to 2016 by Specific Purpose Software GmbH. It is now open sourced, but mostly exists in the form of code fragments.\\
This file is in the Stopeight repository on Github. Please edit here:\\
\href{https://github.com/specpose/stopeight/tree/master/doc/latex/Stopeight_Analyzer.tex}{https://github.com/specpose/stopeight/tree/master/doc/latex/Stopeight\_Analyzer.tex}
A brief overview can be found here:\\
\href{https://www.stopeight.com/dev_analyzer.html}{https://www.stopeight.com/dev\_analyzer.html}\\\\
The Arcs $A \subset \mathbb{R}^2$ are an approximation of a set$\footnote{Midpoints have to be formally included for insertion only}$ $X \subset \mathbb{R}^2$:
\begin{align}
f \circ g: X \rightarrow A
\end{align}
\begin{equation*}
1 < card(X) < \infty
\end{equation*}
via the intermediate functions $f,g$ where the intermediate transversality $Y = X \cap A$ defines Arcs $U_{m}$. ~\cite[\nopp 9.9]{Loring} ($f \circ g$ is neither injective nor surjective) ~\cite[\nopp 2.1]{LauresSzymik}:
\begin{align}
f: X \rightarrow Y; g: Y \rightarrow A
\end{align}
Note: The set of Arcs A is a dim 1-manifold. In $A$ we can find a system of open sets. The set of charts $\{U_{m}\}$ covering the whole manifold $A$ is an atlas. ~\cite[\nopp 3.1.1.]{Fomenko} ~\cite[\nopp 4.5]{Wall} ($X$ is formally a discrete / dim 0-manifold)
\begin{align}
A = \cup_{m}U_{m}; (X = \cup_{n}T_{n})\\
Y: \text{0-submanifold} \Rightarrow dim(X)=0<codim(Y)=1 \Rightarrow X \text{ disjoint } A
\end{align}
\iffalse
Within $f$ we are assigning a sequence of three points to compact Hausdorf invervals ~\cite[\nopp 6.1.3.]{Mortad}:
\begin{equation}
\{x_{n}\}_{n \in \mathbb{N}} \mapsto \{S,C,E\}
\end{equation}\\
\fi
The approximation aims to find the maximal smooth atlas. Each chart $U_{m}$ has a start of Turn $S = \inf_{U_{m}}$, a Corner $C \in U_{m}$ and an end of Turn $E = \sup_{U_{m}}$. This format has the benefit of reducing the amount of data for $Computation$ (see Stopeight Comparator), while preserving the $Representation$ using quadratic bezier splines. ($U_{m}$ closed?)

\section{Representation}

The affine transformations $q$ are functions that map to control points $Q \subset \mathbb{R}^2$:
\begin{align}
q: S \times C \times E \rightarrow Q\\
Q \cap X = 0; Q \cap Y = 0
\end{align}
The combination $M = Y \cup Q$ has parts from the original $X$'s, and constructed $Q$'s that are not in $A$ $\footnote{insertion: $C \notin X$, otherwise: $C \in X \cap A$}$:
\begin{align*}
S,C,E \in X; Q \setminus A=\{q_{1}(S,C,E),q_{3}(S,C,E),q_{4}(S,C,E),q_{6}(S,C,E)\}
\end{align*}
and is composed of four quadratic bezier splines $\{c_{1},...,c_{4}\}$ per chart of M:
\begin{align*}
c_{1}=\{S,q_{1}(S,C,E),q_{2}(S,C,E)\}\\
c_{2}=\{q_{2}(S,C,E),q_{3}(S,C,E),C\}\\
c_{3}=\{C,q_{4}(S,C,E),q_{5}(S,C,E)\}\\
c_{4}=\{q_{5}(S,C,E),q_{6}(S,C,E),E\}
\end{align*}
The Arc segment defining function $arc$ forms the coordinates $a=\{x,y\}$ of Arc $U_{m}$:
\begin{align}
arc: M \times M \times M \times M \rightarrow A\\
(c_{1},c_{2},c_{3},c_{4}) \mapsto qbs(c_{1}) \cup qbs(c_{2}) \cup qbs(c_{3}) \cup qbs(c_{4})
\end{align}
\begin{align}
qbs(c): \mathbb{R} \rightarrow A; (t) \mapsto (x,y); t \in [0,1]
\end{align}
Note: The continuous function $arc$ makes the topological space $A$ path-connected. Therefore it is a $path$ from $U_{m}$ to $U_{m+1}$ and intersects at least once at the joint/unilateral limit $S \cap E$. ($dim (U_{m} \cup U_{m+1}) \geq 1?)$. ~\cite[\nopp 6.1.3.]{Mortad}\\\\
In order to compare the length $L$ from $[a,b] \subseteq A$
\begin{equation}
L(qbs) = \int_{a}^{b}\lvert \frac{\mathrm{d}qbs}{\mathrm{d}t}\rvert \mathrm{d}t
\end{equation}
with $[a.b] \subseteq X \text{discrete}$, we have to connect all $P_{n}, P_{n+1} \in X$ with linear bezier splines:
\begin{align}
lines: X \times X \rightarrow B\\
c_{1}=\{P_{n},\frac{P_{n}+P_{n+1}}{2},P_{n+1}\} \mapsto qbs(c_{1}): \text{Straight line}\\
dim(B)=1<codim(Y)=1 \Rightarrow B \text{ joint } A
\end{align}
Note: For both $B$ and $A$, the length of the curve is now independent of the choice of parameter on the curve. ~\cite[\nopp 1.1]{Taimanov}\\\\
We can now establish an injective $link$ between elements of $B$ and $A$. ($X \rightarrow A$ is injective, $but$ $A \rightarrow X$ is not accurate):
\begin{equation}
link: B \rightarrow A; link^{-1}: A \rightarrow B
\end{equation}
\section{Computation}

\iffalse
\begin{equation} 
\forall u,v \in V :
d(u,v) = 
\begin{cases}
0,  u=v \\
1,  u \neq v 
\end{cases}
\end{equation}
\fi

\printbibliography

\end{document}