\documentclass{report}
\usepackage{amsfonts}
\usepackage{amsmath}
\usepackage{amssymb}
\usepackage{hyperref}

\iffalse
\usepackage{biblatex}
\addbibresource{Stopeight.bib}
\fi
\usepackage{natbib}

\renewcommand{\baselinestretch}{1.25}
\newcommand\norm[1]{\left\lVert#1\right\rVert}

\begin{document}
\title{Not Used Formula}
\author{Fassio Blatter}
\maketitle

\chapter{Introduction}
Within $f$ we are assigning a sequence of three points to compact Hausdorf invervals ~\cite[6.1.3.]{Mortad}:
\begin{equation}
\{x_{n}\}_{n \in \mathbb{N}} \mapsto \{S,C,E\}
\end{equation}\\\\

The purpose of the Analyzer is to find intervals in a Vector Graph and produce an approximation. The input Vector Graph may be obtained from the Grapher, or it may be present in another mathematical context. Certain properties, such as the maximum derivative or the quantity or quality of measure spaces involved, are inherited from this context.\\\\

Non-recursive sections are not proper section types, but merely an expression of intermediate calculation and are therefore of medium computational priority (See discussion of oriented submanifolds in Compact Covers). They do induce a shift inside the Compact Cover if they're not centered.\\\\
The $Max$ thresholds are reductions of measure spaces. The $derivatives$ are affected by $Max_{Straight}$. The Grapher $scale$ interacts with $Max_{Curvre}.$\\\\

Measure Space: Curvature is no more than $(x')^2 + (y')^2 =1$  under any linear transformation of any segment and implies a creation of measure-spaces depending on scale.

\subsection*{Half Cliff}
Even though Half Cliffs are not being used, it is worth mentioning, that they produce Corners just to make the notation complete.
\begin{align}
Max_{Quarter}=\frac{((\pi-2) r^2) /4}{r}\\
C_{3} \ni b'
\end{align}\\\\

Measure Space: Straightness is no more than what is required for a hyperbola not to be considered straight under any linear transformation of any segment and implies a creation of measure spaces depending on intervals of changing signs.\\\\

\section{Recursive Sections}
Derivatives are opening up interval measure spaces. The scale-space sets the maximal order of the polynomials considered. In the method discussed in this paper, it is quartic with restrictions, i.e. the representation is quadratic.  (The functions $q$ and the points $S,C,E$ pose a constraint on the quartic analytic form ~\cite[Spline\_CharacteristicPolynomials.nb]{Stopeight} and its inverse.)\\\\
The following recursive subsections reveal hidden derivatives.

\subsection{Spike}
(Also?)
\begin{align}
angle(a,b',c)\geq\\
straightness(a,b)\approx 1\\
curvature(a,b)>0
\end{align}

\subsection{Relative Areas}
The $solidangle(a,c)$ is defined by the Spike Margin $[a,b',b'',c]$.
\begin{align}
solidangle_{Rel}(a,c) = \frac{area(b',b'')}{(diameter_{horiz}(a,c)/2)^2}
\end{align}

\subsection{Spiral}
Note: $b'$ is the Corner $C_{4}$ of the Eliptic Approximation if the Focal Points overlap (full circle). See Spline Axioms / Eliptic Approximation.

\subsection{ZigZag}
Note: $b'$ is the Corner $C$ of the Eliptic Approximation where the Focal Point is perpendicular to the tangent in $C$, but infinitely far away. See Spline Axioms / Eliptic Approximation.

\chapter{Spectral Separation}
\section{Indicator}
Every zero crossing $t_{m}$ of the 2nd derivative
\begin{equation}
\{t_{m}\}_{m \in \mathbb{Z} , f''(t_{m})=0}
\end{equation}
For all the Intervals $t_{m}-t_{m-1}=T$ of a signal $f: \mathbb{R} \rightarrow \mathbb{R}$
\begin{equation}
\eta(T)=\{t \in \mathbb{R}; t_{m} - t_{m-1}=T \vert f''(t_{m})=0, f''(t_{m-1})=0, t \le t_{m}, t \ge t_{m-1} \}
\end{equation}
Note: $\eta$ is a Lebesgue.\\\\
For all the positive and negative function values $f(t)$ respectively
\begin{align}
\eta_{+}(T)=\{{\eta \vert f(t) \ge 0}\}\\
\eta_{-}(T)=\{{\eta \vert f(t) \le 0}\}
\end{align}
Parseval's theorem
\begin{equation}
\int\limits_{-\infty}^{\infty} \vert f(t) \vert ^2 \mathrm{d} t = 1/2\pi * \int\limits_{-\infty}^{\infty} \vert \hat{f}(1/T) \vert ^2  \mathrm{d} T
\end{equation}
Can be interpreted for signed function values as
\begin{equation}
\int\limits_{-\infty}^{\infty} (\underbrace{\int\limits_{-\infty}^{\infty} \delta (U - T) \mathrm{d} U}_{1} * (\int\int f(t) \mathrm{d} t \underbrace{\mathrm{d} \eta}_{\text{depends on T}}) )\mathrm{d} T = 1/2\pi * \int\limits_{-\infty}^{\infty} \hat{f}(1/T) \mathrm{d} T
\end{equation}
Note: Major $U$ is a unique period, not a point in time. $U$ is not continuous, $\delta$ is the discrete Dirac Delta.\\\\
The convolution of the signal with a periodicity T is equal the distributed energy of two semi-pulse functions of the same periodicity. The negative part $\eta_{-}$ is interpreted as a purely imaginary complex number $z_{2}$.
\begin{equation}
\int\limits_{-\infty}^{\infty}(\int\limits_{-\infty}^{\infty} \delta (U - T) \mathrm{d} U * (\underbrace{\int \int f(t) \mathrm{d} t \mathrm{d} \eta_{+}}_{z_{1}} * \underbrace{\mathrm{i} \int \int f(t) \mathrm{d} t \mathrm{d} \eta_{-}}_{z_{2}}))  \mathrm{d} T = 1/2\pi * \int\limits_{-\infty}^{\infty} ( \int\limits_{-\infty}^{\infty}  f(t) * (cos(t/T)+\mathrm{i} sin(t/T)) \mathrm{d} t ) \mathrm{d} T
\end{equation}
Note: If the angle $\phi$ between two complex numbers $z_{1},z_{2}$ is $\pm\frac{pi}{2}$, their complex multiple $z_{1}*z_{2}$ is equal the real $\lvert z_{1}\rvert*\lvert z_{2}\rvert$. If $x_{1}=y_{2}$, this simplifies to $\lvert x_{1} \rvert ^2$, which is what we have noted above (Parseval Theorem) for the real function $f(t)$ .\\\\
This equation holds for every signal with frequency components separated in the time domain and symmetric in $mod(n/2)=0$ equal length parts. If this is not the case, we have either:\\
1. Longer wavelength pulses overlapping the frequency component, which results in a amplitude modulation of the combined signal.\\
2. Shorter wavelength pulses intersecting the segments of the frequency component, which results in frequency modulation of the combined signal.\\
3. Non-symmetric waveforms and/or different length half-pulses, which can not be removed with this procedure. See Source Separation for further discussion of such signals.\\\\
(A longer wavelength, undetected harmonic can be excluded in this scenario since source separated harmonics are symmetric pulses of integer multiples of the wavelength and we imply that the shortest $symmetric$ pulse is found beforehand in the removal procedure below, i.e. the higher harmonics are $not symmetric$ because of criteria 1)\\\\
Conclusion: This method provides an indication for the spectral separation of the signal. For blind removal of individual pulses, the $waveform$ of the pulse needs to be known. Longer wavelength pulses have the highest likelyhood to contain pulses which violate the above criteria, so their $amplitude$ can not be determined. We have to start removal with the shortest isolated pulses which can be found in the sample.\\\\
The main drawback of Fourier Transformation is that frames (integration bounds) and frequency window (measure) have to be adjusted. There are methods for finding bases in order to remove functions from the combined signal. Fourier Transformation does fulfil the task of spectral separation, but arbitrary precision in the time and frequency domain can not be found.


\iffalse
\printbibliography
\fi
\bibliography{Stopeight}{}
\bibliographystyle{plain}

\end{document}