\documentclass{report}
\usepackage{amsfonts}
\usepackage{amsmath}
\usepackage{amssymb}
\usepackage{hyperref}

\iffalse
\usepackage{biblatex}
\addbibresource{Stopeight.bib}
\fi
\usepackage{natbib}

\renewcommand{\baselinestretch}{1.25}
\newcommand\norm[1]{\left\lVert#1\right\rVert}

\begin{document}
\title{Not Used Formula}
\author{Fassio Blatter}
\maketitle

\chapter{Introduction}
Within $f$ we are assigning a sequence of three points to compact Hausdorf invervals ~\cite[6.1.3.]{Mortad}:
\begin{equation}
\{x_{n}\}_{n \in \mathbb{N}} \mapsto \{S,C,E\}
\end{equation}\\\\

The purpose of the Analyzer is to find intervals in a Vector Graph and produce an approximation. The input Vector Graph may be obtained from the Grapher, or it may be present in another mathematical context. Certain properties, such as the maximum derivative or the quantity or quality of measure spaces involved, are inherited from this context.\\\\

Non-recursive sections are not proper section types, but merely an expression of intermediate calculation and are therefore of medium computational priority (See discussion of oriented submanifolds in Compact Covers). They do induce a shift inside the Compact Cover if they're not centered.\\\\
The $Max$ thresholds are reductions of measure spaces. The $derivatives$ are affected by $Max_{Straight}$. The Grapher $scale$ interacts with $Max_{Curvre}.$\\\\

Measure Space: Curvature is no more than $(x')^2 + (y')^2 =1$  under any linear transformation of any segment and implies a creation of measure-spaces depending on scale.

\subsection*{Half Cliff}
Even though Half Cliffs are not being used, it is worth mentioning, that they produce Corners just to make the notation complete.
\begin{align}
Max_{Quarter}=\frac{((\pi-2) r^2) /4}{r}\\
C_{3} \ni b'
\end{align}\\\\

Measure Space: Straightness is no more than what is required for a hyperbola not to be considered straight under any linear transformation of any segment and implies a creation of measure spaces depending on intervals of changing signs.\\\\

\section{Recursive Sections}
Derivatives are opening up interval measure spaces. The scale-space sets the maximal order of the polynomials considered. In the method discussed in this paper, it is quartic with restrictions, i.e. the representation is quadratic.  (The functions $q$ and the points $S,C,E$ pose a constraint on the quartic analytic form ~\cite[Spline\_CharacteristicPolynomials.nb]{Stopeight} and its inverse.)\\\\
The following recursive subsections reveal hidden derivatives.

\subsection{Spike}
(Also?)
\begin{align}
angle(a,b',c)\geq\\
straightness(a,b)\approx 1\\
curvature(a,b)>0
\end{align}

\subsection{Relative Areas}
The $solidangle(a,c)$ is defined by the Spike Margin $[a,b',b'',c]$.
\begin{align}
solidangle_{Rel}(a,c) = \frac{area(b',b'')}{(diameter_{horiz}(a,c)/2)^2}
\end{align}

\iffalse
\printbibliography
\fi
\bibliography{Stopeight}{}
\bibliographystyle{plain}

\end{document}