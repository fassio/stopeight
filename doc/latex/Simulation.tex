\documentclass{report}
\usepackage{amsfonts}
\usepackage{amsmath}
\usepackage{amssymb}
\usepackage{hyperref}

\iffalse
\usepackage{biblatex}
\addbibresource{Stopeight.bib}
\fi
\usepackage{natbib}

\newcommand\norm[1]{\left\lVert#1\right\rVert}
\renewcommand{\baselinestretch}{1.25}
\begin{document}
\title{Simulation}
\author{Fassio Blatter}
\maketitle

\chapter{Simulation}
(An attempt to extend the algebra to generic problems in physics.)\\\\
Differential Equations?

\section{Wave Trains}
The mathematically most intriguing oriented submanifolds are Crests and Spikes. We introduce a mechanism for the annihilation and creation of pulses. Physical phenomena such as the creation of enthalpy and enthropy during phase transitions could be modeled. A Crest would appear in the outgoing phase when a signal of enthropy brakes in the corresponding medium of the carrier. A Spike is a beat that appears in joint states of crystal grids when the incoming phase accumulates in a $different$ medium.
Traveling wave trains are moving over stationary reef chains. Upon creation, enthropic wave trains are impulses, at the least even a part of a half-pulse (Example Spiral). Enthalpic reef chains change state when they are converted to a plasma state for an infinitesimal short period of time.\\\\
A wave train appears in a wave packet that splits up. A wave packet that splits, decreases its isolation and transfers a part of its visibility to the sub-wave packets. Depending on the wavelength of a Reef, the original wave packet may never appear again.\\\\
On the other hand, the sub-wave packets of an open ocean wave pick up wind. The enclosing wave packet can form a new wave train, together with neighbouring wave packets and so slowly, the little left-over, chaotic visibility of the sub-wave packets gets transferred not only to the enclosing wave packet, but to the whole wave train. The swell gets groomed and the short wavelengths disappear.\\\\
Because in any accurate measurement of matter, only a limited number of reef chains can be considered; For the multitude of periodic energy passing over a reef chain, all products of the chain are compared, up to the length of the frame of the train.
It could make sense to map such phenomena in a complex plane. Could we explain why for example a Crest starts appearing on a Reef, and different, purely imaginary waveforms appear in open ocean waters where waves travel in arbitrary directions.

\subsection{Wave Generation}
For Dune/Crest or Swing (stationary wave).\\
Supraliquid\\
Sail Effect / Surface Tension

\subsection{Phase Transitions}
For Dune/Spiral.
\begin{align}
z =  \sup_{U}\lvert curvature([a,b]) \rvert*(\cos{(isolation([a,b]))} +\mathrm{i} \sin{(visibility([a,b]))})
\end{align}

\subsection{Lightning}
For Spike/ZigZag.\\
Superconductor\\
If we're setting $\sqrt{1-y^2}$ to be the linear part of complex plane, $\phi$ in $z=\sqrt{1-y^2}e^{i\phi}$ is the rotational part.
\begin{align}
z =  straightness([a,b]) + \mathrm{i} curvature([a,b])
\end{align}

\subsection{Stateless}
Supercritical\\
gaseous/liquid carbondioxide

\iffalse
\printbibliography
\fi
\bibliography{Stopeight}{}
\bibliographystyle{plain}

\end{document}